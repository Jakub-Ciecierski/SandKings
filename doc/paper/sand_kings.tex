
\documentclass[conference]{IEEEtran}
%\documentclass[a4paper]{IEEEconf}

% This package serves to balance the column lengths on the last page of the document.
% please, insert \balance command in the left column of the last page
\usepackage{balance}

%% to enable \thank command
\IEEEoverridecommandlockouts 
%% The usage of the following packages is recommended
%% to insert graphics
\usepackage[dvips]{graphicx}
% to typeset algorithms
\usepackage{algorithmic}
\usepackage{algorithm}
% to typeset code fragments
\usepackage{listings}
% to make an accent \k be available
\usepackage[OT4,T1]{fontenc}
% provides various features to facilitate writing math formulas and to improve the typographical quality of their output.
\usepackage[cmex10]{amsmath}
\interdisplaylinepenalty=2500
% por urls typesetting and breaking
\usepackage{url}
% for vertical merging table cells
\usepackage{multirow}

% ADD NEW PACKAGES HERE:




% define environments for remarks and examples
\newtheorem{remark}{Remark}[section]
\newtheorem{example}[remark]{Example}


%
%
\title{Multi-agent based simulation of G. R. R. Martin's Sand Kings}
%
%
\author{
\IEEEauthorblockN{Jakub Ciecierski, Viet Ba Mai, \\
					Michal Slupczynski and Wojciech Zyskowski}
					
\IEEEauthorblockA{Faculty of Mathematics and Information Science,\\ 
					Warsaw University of Technology \\
					Plac Politechniki 1, 00-660 Warsaw, Poland}
}


\begin{document}
\maketitle







% ******************************* <VB> ******************************* %

%-------------------------------------------------------------------
%-------------------------------------------------------------------
%-------------------------------------------------------------------
\begin{abstract}

The abstract about this document


\end{abstract}

% ******************************* </VB> ******************************* %








% ******************************* <JC> ******************************* %

%-------------------------------------------------------------------
%-------------------------------------------------------------------
%-------------------------------------------------------------------
\section{Introduction}


\IEEEoverridecommandlockouts\IEEEPARstart{T}{he fancy} Introduction. Motivation, Objective, The book. Maybe structure

% ******************************* </JC> ******************************* %







% ******************************* <WZ> ******************************* %

%-------------------------------------------------------------------
%-------------------------------------------------------------------
%-------------------------------------------------------------------
\section{Tools}

Repast, what why etc.


% ******************************* </WZ> ******************************* %









%-------------------------------------------------------------------
%-------------------------------------------------------------------
%-------------------------------------------------------------------
\section{Methodology}
Small introduction to methodology






% ******************************* <MS + VB> ******************************* %


%-------------------------------------------------------------------
%-------------------------------------------------------------------
\subsection{Agents}
What is an agent. Mobiles, Maw, God, Monsters etc.



%-------------------------------------------------------------------
%-------------------------------------------------------------------
\subsection{Environment}
Map, food


% ******************************* </MS + VB> ******************************* %







% ******************************* <WZ> ******************************* %


%-------------------------------------------------------------------
%-------------------------------------------------------------------
\subsection{Desire System}
Things about the desire system


% ******************************* </WZ> ******************************* %






% ******************************* <JC> ******************************* %

%-------------------------------------------------------------------
%-------------------------------------------------------------------
\subsection{Communication}
Things about knowledge, communication etc.

% ******************************* </JC> ******************************* %





% ******************************* <ALL> ******************************* %

%-------------------------------------------------------------------
%-------------------------------------------------------------------
%-------------------------------------------------------------------
\section{Simulation}
Small intro to simulation. What are we planning to do in this section.

% ******************************* </ALL> ******************************* %







% ******************************* <MS> ******************************* %

%-------------------------------------------------------------------
%-------------------------------------------------------------------
\subsection{Data set}
What do we have in simulation. Environment, maws, mobiles, how many etc.

% ******************************* </MS> ******************************* %







% ******************************* <ALL> ******************************* %

%-------------------------------------------------------------------
%-------------------------------------------------------------------
\subsection{Results}
Results of our simulation.

% ******************************* </ALL> ******************************* %






% ******************************* <ALL> ******************************* %

%-------------------------------------------------------------------
%-------------------------------------------------------------------
%-------------------------------------------------------------------
\section{Conclusions}
Conclusions of this paper.

% ******************************* </ALL> ******************************* %










% serves to balance the column lengths on the last page of the document
% should be inserted the left column of the last page
\balance




\section*{Appendix}
Is it even needed ?

\section*{Acknowledgment}
Some thanks to Paprzycki ?


\begin{thebibliography}{99}

\bibitem{c1}
Source example, for more add sources here

\end{thebibliography}


\end{document}
